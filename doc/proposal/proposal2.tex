%%% Local Variables:
%%% mode: latex
%%% Tex-master: t
%%% LaTeX-command: "pdflatex -synctex=1 -aux-directory auto %s"
%%% End:
\documentclass{article}

\addtolength{\hoffset}{-2.25cm}
\addtolength{\textwidth}{4.5cm}
\addtolength{\voffset}{-3.25cm}
\addtolength{\textheight}{5cm}
\setlength{\parskip}{6pt}
\setlength{\parindent}{0in}

%----------------------------------------------------------------------------------------
%	PACKAGES AND OTHER DOCUMENT CONFIGURATIONS
%----------------------------------------------------------------------------------------

% \usepackage{blindtext} % Package to generate dummy text
\usepackage[utf8]{inputenc} % Use UTF-8 encoding
\usepackage{microtype} % Slightly tweak font spacing for aesthetics
\usepackage[english]{babel} % Language hyphenation and typographical rules
\usepackage{amsthm, amsmath, amssymb, bm} % Mathematical typesetting
% \usepackage{float} % Improved interface for floating objects
\usepackage{graphicx, multicol} % Enhanced support for graphics
\usepackage{xcolor} % Driver-independent color extensions
\usepackage{booktabs} % Enhances quality of tables
% \usepackage{csquotes} % Context sensitive quotation facilities
\usepackage[yyyymmdd]{datetime} % Uses YEAR-MONTH-DAY format for dates
\renewcommand{\dateseparator}{-} % Sets dateseparator to '-'
\usepackage{fancyhdr} % Headers and footers
\usepackage{enumerate}
\pagestyle{fancy} % All pages have headers and footers
\fancyhead{}\renewcommand{\headrulewidth}{0pt} % Blank out the default header

\fancyfoot[L]{} % Custom footer text
\fancyfoot[C]{} % Custom footer text
\fancyfoot[R]{\thepage} % Custom footer text

\newcommand{\note}[1]{\marginpar{\scriptsize \textcolor{red}{#1}}} % Enables comments in red on margin

%----------------------------------------------------------------------------------------

\newcommand{\cP}{\mathcal{P}}
\newcommand{\cX}{\mathcal{X}}
\newcommand{\cL}{\mathcal{L}}

\newcommand{\R}{\mathbb{R}}
\newcommand{\E}{\mathbb{E}}

\newcommand{\simiid}{\overset{\text{i.i.d.}}{\sim}}

\newcommand{\nodeord}{\boldsymbol{\prec}}
\newcommand{\fnr}{\epsilon^-}
\newcommand{\fpr}{\epsilon^+}
\newcommand{\mutr}{\mu}

\begin{document}

\fancyhead[C]{}
\hrule \medskip % Upper rule

\begin{minipage}{0.295\textwidth}
  \raggedright{} \small
  Maxwell Murphy \hfill
  murphy2122@berkeley.edu \hfill
\end{minipage}
\begin{minipage}{0.4\textwidth}
  \centering \large Transmission Network Bayesian Inference
\end{minipage}
\begin{minipage}{0.295\textwidth}
  \raggedleft{} \today \hfill
\end{minipage}

\medskip \hrule % Lower rule
\bigskip

\section{Transmission Network Simulations}
Would like to control generating networks with the following parameters

Parameters:
\begin{itemize}
  \item Reproduction coeff (expected branching factor) $\rightarrow$ as a surface?
  \item Founder proportion
  \item Observed proportion
  \item Superinfection rate?
\end{itemize}

Huber transmission network model appears to consider random networks with epidemiological parameters overlaid. SIMPLEGEN takes epidemiological parameters and generates a transmission record.

Initial plan: generate networks a la Huber with overlaid epidemiological parameters.

\section{Transmission Process Model}
Several components possible
\begin{itemize}
  \item Spatial --- Diffusion process, other models of connectivity, quantification of contact rate between individuals
  \item Temporal --- Serial interval models, considering the symptomatic vs asymptomatic infections
  \item Behavioral Risk --- Travel
  \item Latent Genetics --- Model of malaria biology, representation reflects the type of assay used, simple model of allele retention and mutation, or direct haplotype modeling?
  \item Augment with Superinfection --- Straight forward for spatial, temporal, less straight forward for genetics unless we treat latent genetics as separable (direct haplotype inference) otherwise consider all possible allele origins for a parent set. One thing that is nice is that including epidemiological components will naturally regularize the likelihood. Genetics only will suffer from model inflation.
\end{itemize}

A challenge of this approach is dealing with the ``unobserved'' infections. When we say a node comes from the source population, we're saying that the distance between this node and any other observed node is greater than some $n$ generations. How do we tune this $n$?

\section{Observation Process Model}
Observation process based on knowledge about the assay. Sequencing based technology will have different dynamics versus say microsatellites or snps

\section{Random Thoughts}

\begin{itemize}
  \item Individual level Rc is directly calculable from the ordering
  \item Is there something we can say re: how many observations we make? i.e.\ if we observe 20\% of all nodes at random, what are our bounds on Rc? How does our sampling scheme bias this? Could be that when we pursue related infections, are we biasing our estimates?
  \item MERFAT, look up data and talk to sofonias
  \item jamie lloyd smith
\end{itemize}

\newpage

\section{Transmission model}
\begin{itemize}
  \item Distribute founder population according to some dist $P_f(\mu, \pi, G_{f})$ where $\mu$ is the mean complexity of infection, $\pi$ are the allele frequencies, and $G_{f}$ is the spatial density of founders.
  \item Forward local transmission nodes are modeled as an inhomogeneous spatiotemporal poisson process centered around a given parent node.
        \begin{itemize}
          \item $\Lambda$ as a measure of transmission capacity
          \item self-exciting point process?
        \end{itemize}
\end{itemize}


\end{document}
